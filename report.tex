\documentclass{data/hbue}
\begin{document}

%导入基本信息
%用来生成论文封面和开题报告封面信息

%论文{题目}
\hbuetitle{局域网大文件并发传输设计与实现} 

%学生{姓名}
\hbueauthor{夏彬}

%指导老师{姓名}{职称}
\hbuementor{宋莺}{讲师} 

%{所属院校}{专业}{年级}{班级}{学号}
\hbueinfo{信息管理学院}{计算机科学与技术}{S1021}{2010}{1032149}

%{开题报告时间}
\hbuereporttime{2011年1月1日}


%!!!以下内容均直接插入table之内因此不要在内容里面使用\\换行 
%需要换行直接使用空白行
%
%一、题目来源及研究的理论与实际意义
\reportabstract
{
\begin{description}
	\item{题目来源:} 
		5年的linux长期使用经验以及多次linux下编写程序在windows、linux运行的
		经验,更多的是通过Internet向优秀的Hacker们学习的经历使我萌生此想法以
		此毕业论文来总结我大学生涯。

		用来贯穿论文的实例是关于---局域网大文件并发传输的研究。来源与在
		校期间上实验课时需要根据课程要求安装特定软件,但一般软件都比较大,
		虽然内网下载速度很快,但由于需要几十人同时下载致使平局速度急剧下降。

	\item{意义:}
		在桌面计算机上主流的三种操作系统有linux、Mac和Windows。
		国内Windows在过去占据着主导位置,随着苹果这几年的崛起国内IT公司已经开
		始注重其他系统平台用户。现在国内很多软件都已经拥有Mac版本,少部分软件
		拥有linux版本。而且手机、平板这些便携式电脑已经越来越流行,软件平台早
		已不再局限在Windows上。 也不会局限在某一特定事物之上。
		
		国内很多人特别是大学生因为没有经验容易被局限在某一特定软件平台、特定
		编程语言、特定工具之下。 而且随着时间推移会越来越依赖自己所熟悉的而
		无法去体会其他精彩的东西。我希望通过本篇论文抛砖引玉,尽我所能让师弟
		师妹们能去领略一下计算机革命的精彩。

\end{description}
}

%二、国内外相关研究成果及研究动态描述
\reportdescription
{
随着网络基础设施的逐渐完善,下载文件已经是越来越普遍需求也越来越高
国内有名的普通下载软件有早期的快车,最近几年流行的迅雷以及腾讯的旋风。
局域网内下载软件却很少有特别出名的,
国外比较有名的局域网传输软件有日本的飞鸽传输,美国的HFS(Http File Server)。
国内局域网一般都是通过架设FTP,飞鸽传书,QQ以及Windows自带的网络邻居来下载。
其中FTP方式和Windows网络邻居在传输大量小文件时准备时间过长而且这
两种方式都需要一定的配置才能运行。
QQ方式需要有外网连接,但有时会出现判断错误导致文件通过外网进行
传输。
飞鸽传书在局域网内利用UDP进行文件传输速度非常快。
}

%三、主要研究内容  
\reportcontent
{
在linux下(同样适应与windows、Mac、Unix等)进行软件开发的相关事宜。
使用开源工具构建整个软件系统。
\begin{itemize}
	\item 通过Umbrello进行UML设计,Dia绘制其他示意图。
	\item 通过vim编写程序,latex编写文档。
	\item 通过CMake进行自动编译,使用git管理项目。
	\item 通过Inkscape制作矢量图形,GIMP绘制位图来提高程序界面的友好度。
	\item 通过GCC进行跨平台交叉编译。
	\item 通过VirtualBox测试其他平台运行效果。
	\item 以及最重要的通过各个阶段的开发来介绍优秀的工具。
\end{itemize}

对于软件本身需要重点设计的有
\begin{itemize}
	\item 文件管理
		\begin{itemize}
			\item 文件按照什么规则进行分开以便传输。
			\item 文件如何进行完整性检验。
			\item 文件信息如何组织以便其他节点知晓。
			\item 文件的重组。
		\end{itemize}
	\item 传输协议
		\begin{itemize}
			\item 如何发现网络内其他用户。
			\item 如何控制传输速度防止过多的丢包。
			\item 如何让
		\end{itemize}
\end{itemize}

}

%四、本研究的关键创新点,研究方法、研究进度计划和完成时间 
\reportschedule
{
创新点:
本文通过一个实例---局域网内大文件并发传输来介绍开源世界里的某些工作方式。
实例本身也有一定特色,
\begin{itemize}
	\item 一般下载软件都受限于带宽限制, 本系根据特定情况(局域网内多
		人下载同一文件)下合理使用UDP广播方式进行文件传输``突破了''
		带宽限制。
	\item 对于带宽充足的情况下,比如两台千兆网卡的计算机使用网线直连进行文件
		传输则传输瓶颈受限于硬盘写速度(时下典型情况30,40MB/s)。本系统使用
		内存影射利用大内存(配置千兆网卡的计算机都会有2-4G的内存)来进行缓冲
		从而突破硬盘读写速度限制。
\end{itemize}

研究方法: 
通过从软件设计到编写代码到编译部署的实际操作来完成本论文。
研究进度计划和完成时间:
}

%五、主要参考文献
\reportreference
{
各大文库阿
}

%六、指导老师意见
\reportsuggestion
{
}

%导入开题报告的实际排版文件
\pagestyle{empty}
\newcommand{\ulbox}[1]{\ul{\makebox[10cm]{#1}}}

\begin{center}
	\renewcommand{\baselinestretch}{3.5}
	{\xiaoer\heiti 湖北经济学院本科毕业生毕业论文\-(毕业设计)} \\[1cm]
	{\chuhao\bfseries 开~题~报~告~书} \\[5cm]

	\sanhao
	\begin{tabular}{lc} 
		题\qquad 目:& \ulbox{\kaiti\stitle}  \\ 
		指导老师:& \ulbox{\kaiti\smentorn\-(\smentort)} \\
		学生姓名:& \ulbox{\kaiti\sauthor} \\
		所属院系:& \ulbox{\kaiti\sinfos} \\
		专业名称:& \ulbox{\kaiti\sinfop} \\
		班级名称:& \ulbox{\kaiti\sinfoc} \\
		开题时间:& \ulbox{\kaiti\sreporttime} \\	
	\end{tabular}
\end{center}
	\renewcommand{\baselinestretch}{1.5}

\newpage

{\sihao\center 湖北经济学院本科毕业生毕业论文\-(毕业设计)\qquad 开题报告书}
\begin{longtable}{|l|c|l|c|l|l|}%
	\hline
	\xiaosi
	姓\quad 名 & {\kaiti\sauthor} &
	专业和班级 & {\kaiti\sinfop ~ \sinfoc} &
	开题时间  &{\kaiti\sreporttime} \\
	\hline
	题\quad 目  & \multicolumn{5}{c|}{\kaiti\stitle} \\\hline

	\multicolumn{6}{|p{\textwidth}|}{
	一、题目来源及研究的理论与实际意义

	\sreportabstract

	\vfill
	}\\\hline

	\multicolumn{6}{|p{\textwidth}|}{
	二、国内外相关研究成功及研究动态描述

	\sreportdescription

	\vfill
	}\\\hline

	\multicolumn{6}{|p{\textwidth}|}{
	三、主要研究内容  

	\sreportcontent 

	\vfill
	}\\\hline

	\multicolumn{6}{|p{\textwidth}|}{
	四、本研究的关键创新点,研究方法、研究进度计划和完成时间 

	\sreportschedule

	\vfill
	}\\\hline

	\multicolumn{6}{|p{\textwidth}|}{
	五、主要参考文献

	\sreportreference 

	\vfill
	}\\\hline

	\multicolumn{6}{|p{\textwidth}|}{
	六、指导老师意见

	\sreportsuggestion

	\vfill
	}\\ \hline
\end{longtable}


\end{document}
