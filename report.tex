\documentclass{data/hbue}
\begin{document}

%导入基本信息
%用来生成论文封面和开题报告封面信息

%论文{题目}
\hbuetitle{局域网大文件并发传输设计与实现} 

%学生{姓名}
\hbueauthor{夏彬}

%指导老师{姓名}{职称}
\hbuementor{宋莺}{讲师} 

%{所属院校}{专业}{年级}{班级}{学号}
\hbueinfo{信息管理学院}{计算机科学与技术}{S1021}{2010}{1032149}

%{开题报告时间}
\hbuereporttime{2011年1月1日}


%!!!以下内容均直接插入table之内因此不要在内容里面使用\\换行 
%需要换行直接使用空白行
%
%一、题目来源及研究的理论与实际意义
\reportabstract
{
\begin{description}
	\item{题目来源:} 

		在校期间曾多次遇到教师上课时需要发送一些文件给学生安装以便更好的进行
		教学安排,但在文件较大时局域网即使处在满负荷之下也无法满足人数众多的
		学生同时下载.

		因此萌生使用UDP的广播特性来进行文件传输这一想法.
		考虑到操作系统的多样化以及自己一直是在linux环境下成长的,因此跨平台是
		这款软件的一个必备要求.

	\item{意义:}

		在桌面计算机上主流的三种操作系统有linux、Mac和Windows。
		国内Windows在过去占据着主导位置,随着苹果这几年的崛起国内IT公司已经开
		始注重其他系统平台用户。现在国内很多软件都已经拥有Mac版本,少部分软件
		拥有linux版本。而且手机、平板这些便携式电脑已经越来越流行,软件平台早
		已不再局限在Windows上。 也不会局限在某一特定事物之上。

		跨平台不仅仅意味着能在多个系统上运行,同时促使软件开发人员在开发软件时
		尽量不用操作系统特性而是使用更具抽象性的工具如C++的boost库.这样能更好
		的进行抽象设计,而不用过早陷入细节问题.

		UDP本身是不适合进行大文件传输的特别是在互联网上, 但合理使用广播或多播
		这种利用以太网本身的传输机制进行合理的利用带宽, 就可以让这种并发传输
		的情况下非常明显的``突破''网络物理限制. 
		可以节约宝贵的时间,特别是在课堂上.
		
\end{description}
}

%二、国内外相关研究成果及研究动态描述
\reportdescription
{
\begin{description}
	\item{国内情况:}

		随着网络基础设施的逐渐完善,下载文件已经是非常普遍的需求,
		国内有名的下载软件有早期的快车,最近几年流行的迅雷以及腾讯的旋风。
		局域网内下载软件由于应用较少却因此少有特别出名的,
		国外比较有名的局域网传输软件有日本的飞鸽传输和美国的
		HFS(Http File Server)。

	\item{国外情况:}

		在中国高校一般都是通过架设FTP,飞鸽传书,QQ以及Windows自带的网络
		邻居来下载.其中FTP方式和Windows网络邻居在传输大量小文件时准备时
		间过长而且这两种方式都需要一定的配置才能运行.QQ方式需要有外网连
		接,而且有时会出现判断错误导致文件通过外网进行传输.飞鸽传书在局
		域网内利用UDP进行文件传输速度非常快, 但还是无法解决文件的并发传
		输.同样利用广播进行文件传输的有网络模式下的ghost,可以在局域网内
		进行批量按照系统. 

\end{description}
}

%三、主要研究内容  
\reportcontent
{
在linux下(同样适应与windows、Mac、Unix等)进行软件开发的相关事宜。
使用开源工具构建整个工具。
\begin{itemize}
	\item 涉及到用Umbrello进行UML设计,Dia绘制其他示意图。
	\item 涉及到用vim编写程序,latex编写文档。
	\item 涉及到用CMake进行自动编译,用git管理项目。
	\item 涉及到用Inkscape制作矢量图形提高程序界面的友好度。
	\item 涉及到用GCC进行跨平台交叉编译。
	\item 涉及到用VirtualBox测试windows平台运行效果。
	\item 涉及到用python编写测试脚本, 生成特定代码.
\end{itemize}

对于工具本身需要重点设计的有
\begin{itemize}
	\item 文件管理
		\begin{itemize}
			\item 文件按照什么规则进行分块以便传输。
			\item 文件如何进行完整性检验。
			\item 文件信息如何组织以便其他节点知晓。
			\item 文件的重组。
		\end{itemize}
	\item 传输方面
		\begin{itemize}
			\item 如何发现网络内其他用户。
			\item 如何控制传输速度防止过多的丢包。
			\item 如何在丢包发生后启动重传机制。
			\item 如何在
		\end{itemize}
	\item 众多GUI库该如何选取.
\end{itemize}

}

%四、本研究的关键创新点,研究方法、研究进度计划和完成时间 
\reportschedule
{
本文通过一个实例---局域网内并发传输来介绍开源界里的一些工作方式

研究方法尽量勇于尝新,以便更好的达到学习效果.
比如在可以的情况下尽量使用编译器还无法完全支持的新标准C++11的一些
新特性来使代码结构更加简单更少的依赖第三方块。
在GUI的选型上也不没有选择自己熟悉的一些库,而是将整个界面构建在WEB
界面之上.

\{研究进度计划和完成时间\}
}

%五、主要参考文献
\reportreference
{
\{参考文献\}
}

%六、指导老师意见
\reportsuggestion
{
}

%导入开题报告的实际排版文件
\pagestyle{empty}
\newcommand{\ulbox}[1]{\ul{\makebox[10cm]{#1}}}

\begin{center}
	\renewcommand{\baselinestretch}{3.5}
	{\xiaoer\heiti 湖北经济学院本科毕业生毕业论文\-(毕业设计)} \\[1cm]
	{\chuhao\bfseries 开~题~报~告~书} \\[5cm]

	\sanhao
	\begin{tabular}{lc} 
		题\qquad 目:& \ulbox{\kaiti\stitle}  \\ 
		指导老师:& \ulbox{\kaiti\smentorn\-(\smentort)} \\
		学生姓名:& \ulbox{\kaiti\sauthor} \\
		所属院系:& \ulbox{\kaiti\sinfos} \\
		专业名称:& \ulbox{\kaiti\sinfop} \\
		班级名称:& \ulbox{\kaiti\sinfoc} \\
		开题时间:& \ulbox{\kaiti\sreporttime} \\	
	\end{tabular}
\end{center}
	\renewcommand{\baselinestretch}{1.5}

\newpage

{\sihao\center 湖北经济学院本科毕业生毕业论文\-(毕业设计)\qquad 开题报告书}
\begin{longtable}{|l|c|l|c|l|l|}%
	\hline
	\xiaosi
	姓\quad 名 & {\kaiti\sauthor} &
	专业和班级 & {\kaiti\sinfop ~ \sinfoc} &
	开题时间  &{\kaiti\sreporttime} \\
	\hline
	题\quad 目  & \multicolumn{5}{c|}{\kaiti\stitle} \\\hline

	\multicolumn{6}{|p{\textwidth}|}{
	一、题目来源及研究的理论与实际意义

	\sreportabstract

	\vfill
	}\\\hline

	\multicolumn{6}{|p{\textwidth}|}{
	二、国内外相关研究成功及研究动态描述

	\sreportdescription

	\vfill
	}\\\hline

	\multicolumn{6}{|p{\textwidth}|}{
	三、主要研究内容  

	\sreportcontent 

	\vfill
	}\\\hline

	\multicolumn{6}{|p{\textwidth}|}{
	四、本研究的关键创新点,研究方法、研究进度计划和完成时间 

	\sreportschedule

	\vfill
	}\\\hline

	\multicolumn{6}{|p{\textwidth}|}{
	五、主要参考文献

	\sreportreference 

	\vfill
	}\\\hline

	\multicolumn{6}{|p{\textwidth}|}{
	六、指导老师意见

	\sreportsuggestion

	\vfill
	}\\ \hline
\end{longtable}


\end{document}
